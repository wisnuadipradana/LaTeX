\documentclass[11pt,a4paper]{article}
\usepackage{geometry}\geometry{tmargin=2cm,lmargin=2cm,bmargin=2cm,rmargin=2cm}
\usepackage[utf8]{inputenc}
\usepackage{amsmath}
\usepackage{amsfonts}
\usepackage{amssymb}

\DeclareMathOperator{\lcm}{$\text{lcm}$}
\newcommand{\ds}{\displaystyle}
\newcommand{\bs}{\boldsymbol}
\usepackage{multicol}


\author{Uzumaki Nagato Tenshou}
\title{Catatan Persamaan Diferensial Parsial (MA5271)}

\usepackage{fancyhdr}
\rhead{Uzumaki Nagato Tenshou}
\lhead{}

\usepackage{amsthm}
\makeatletter
\def\th@plain{%
	\thm@notefont{}% same as heading font
	\itshape % body font
}
\def\th@definition{%
	\thm@notefont{}% same as heading font
	\normalfont % body font
}
\makeatother

\theoremstyle{plain}
\newtheorem{theorem}{Teorema}[section]
\newtheorem{cor}{Akibat}[theorem]
\newtheorem{lemma}[theorem]{Lemma}
\newtheorem{proposition}[theorem]{Proposition}
\newtheorem{conjecture}[theorem]{Conjecture}
\newtheorem{sifat}[theorem]{Sifat}

\theoremstyle{definition}
\newtheorem{defn}[theorem]{Definition}
\newtheorem{conj}[theorem]{Conjecture}
\newtheorem{exmp}[theorem]{Example}
\newtheorem{prob}[theorem]{Problem}

\theoremstyle{remark}
\newtheorem*{remark}{Remark}
\newtheorem*{note}{Note}


\newcommand{\PP}{\mathbb{P}}
\newcommand{\R}{\mathbb{R}}
\newcommand{\LL}{\mathfrak{L}}


\begin{document}
\maketitle
\pagestyle{fancy}

Persamaan diferensial parsial adalah persamaan diferensial bagi fungsi peubah banyak $u(x,y,\cdots)$. Orde dari PDP adalah turunan tertinggi yang muncul pada PDP tersebut. \\
\\
Berbagai persamaan diferensial yang penting
\begin{enumerate}
	\item[1.] $u_{x}+u_{t}=0$ (Persamaan Transport).
	\item[2.] $u_{x}+u u_{t}=0$ (Persamaan Burgers(Shockwave), merupakan bentuk khusus persamaan Buckley-Leverett).
	\item[3.] $u_{t} = k u_{xx}$ (Persamaan panas, Persamaan difusi).
	\item[4.] $u_{tt}=c^{2}u_{xx}$ (Persamaan Gelombang).
	\item[5.] $u_{xx}+u_{yy}=0=\Delta u = 0$ (Persamaan Laplace).
\end{enumerate}



\begin{flushleft}
	\textbf{Operator Linear}
\end{flushleft}
 Suatu operator $\LL$ dikatakan \textit{linier} jika
 $$\LL(u+v)=\LL u+\LL v, \text{ dan } \LL(cu) = c\LL u$$
 untuk setiap fungsi $u,v$ dan untuk setiap $c\in \R$.\\
 Suatu PDP berbentuk $$\LL u=0$$ dikatakan linier jika $\LL$ operator linier.

\begin{enumerate}
	\item[1.] Akan dibuktikan persamaan difusi merupakan PDP linier. \- \\
	\textbf{Penyelesaian.} Diketahui persamaan umum difusi $u_{t}=ku_{xx}$ atau ditulis $$ \partial_{t} u - k \partial_{xx}u = (\partial_{t}-k \partial_{xx})u = 0.$$
	Sehingga dapat dituliskan dalam bentuk $\LL u=0$, dengan operator $\ds \LL = \partial_{t}-k \partial_{xx}$.\\
	Ambil sebarang fungsi $u,v$ dan $c\in \R$ kemudian tinjau
	\begin{align*}
	\LL(u+cv) &=  (\partial_{t}-k \partial_{xx})(u+cv) \\
	&=  (\partial_{t}-k \partial_{xx})u +(\partial_{t}-k \partial_{xx})(cv) \\
	&= \partial_{t}u-k \partial_{xx}u + \partial_{t}(cv)-k \partial_{xx}(cv) \\
	&= \partial_{t}u-k \partial_{xx}u + c \partial_{t}v-k  c \partial_{xx}v \\
	&= \partial_{t}u-k \partial_{xx}u + c(\partial_{t} - k \partial_{xx}v) \\
	&= (\partial_{t}- k \partial_{xx}) u + c(\partial_{t} - k  \partial_{xx}) v \\
	&= \LL u + c\LL v
	\end{align*}
	Hal ini berakibat bahwa persamaan difusi merupakan PDP linier.
	\item[2.] Akan dibuktikan persamaan transport merupakan PDP linier. \- \\
	\textbf{Penyelesaian.} Diketahui persamaan transport $u_{x}=u_{t}$ atau ditulis $$ \partial_{x} u - \partial_{t}u = (\partial_{x}-\partial_{t})u = 0.$$
	Sehingga dapat dituliskan dalam bentuk $\LL u=0$, dengan operator $\ds \LL = \partial_{x}-\partial_{t}$.\\
	Ambil sebarang fungsi $u,v$ dan $c\in \R$ kemudian tinjau 
	\begin{align*}
	\LL(u+cv) &=  (\partial_{x}-\partial_{t})(u+cv) \\
	&=  (\partial_{x}-\partial_{t})u +(\partial_{x}-\partial_{t})(cv) \\
	&= \partial_{x}u-\partial_{t}u + \partial_{x}(cv)-\partial_{t}(cv) \\
	&= \partial_{x}u-\partial_{t}u + c \partial_{x}v - c \partial_{t}v \\
	&= (\partial_{x}- \partial_{t}) u + c(\partial_{x} - \partial_{t}) v \\
	&= \LL u + c\LL v
	\end{align*}
	Hal ini berakibat bahwa persamaan transport merupakan PDP linier.
	\item[3.] Akan dibuktikan persamaan gelombang merupakan PDP linier. \- \\
	\textbf{Penyelesaian.} Diketahui persamaan umum gelombang $u_{tt} - c^{2} u_{xx} = 0$ atau ditulis $$ \partial_{tt} u - c^{2} \partial_{xx}u = (\partial_{tt}-c^{2} \partial_{xx})u = 0.$$
	Sehingga dapat dituliskan dalam bentuk $\LL u=0$, dengan operator $\ds \LL = \partial_{tt}-c^{2} \partial_{xx}$.\\
	Ambil sebarang fungsi $u,v$ dan $k\in \R$ kemudian tinjau
	\begin{align*}
	\LL(u+cv) &=  (\partial_{tt}-c^{2} \partial_{xx})(u+kv) \\
	&=  (\partial_{tt}-c^{2} \partial_{xx})u +(\partial_{tt}-c^{2} \partial_{xx})(kv) \\
	&= \partial_{tt}u-c^{2} \partial_{xx}u + \partial_{tt}(kv)-c^{2} \partial_{xx}(kv) \\
	&= \partial_{tt}u-c^{2} \partial_{xx}u + k \partial_{tt}v-c^{2} k \partial_{xx}v \\
	&= \partial_{tt}u-c^{2} \partial_{xx}u + k(\partial_{tt}v - c^{2} \partial_{xx}v) \\
	&= (\partial_{tt}- c^{2} \partial_{xx}) u + k(\partial_{tt} - c^{2}  \partial_{xx}) v \\
	&= \LL u + k\LL v
	\end{align*}
	Hal ini berakibat bahwa persamaan gelombang merupakan PDP linier. 
	\item[4.] Akan dibuktikan persamaan gelombang merupakan PDP linier. \- \\
	\textbf{Penyelesaian.} Diketahui persamaan umum gelombang $u_{tt} - c^{2} u_{xx} = 0$ atau ditulis $$ \partial_{tt} u - c^{2} \partial_{xx}u = (\partial_{tt}-c^{2} \partial_{xx})u = 0.$$
	Sehingga dapat dituliskan dalam bentuk $\LL u=0$, dengan operator $\ds \LL = \partial_{tt}-c^{2} \partial_{xx}$.\\
	Ambil sebarang fungsi $u,v$ dan $k\in \R$ kemudian tinjau
	\begin{align*}
	\LL(u+cv) &=  (\partial_{tt}-c^{2} \partial_{xx})(u+kv) \\
	&=  (\partial_{tt}-c^{2} \partial_{xx})u +(\partial_{tt}-c^{2} \partial_{xx})(kv) \\
	&= \partial_{tt}u-c^{2} \partial_{xx}u + \partial_{tt}(kv)-c^{2} \partial_{xx}(kv) \\
	&= \partial_{tt}u-c^{2} \partial_{xx}u + k \partial_{tt}v-c^{2} k \partial_{xx}v \\
	&= \partial_{tt}u-c^{2} \partial_{xx}u + k(\partial_{tt}v - c^{2} \partial_{xx}v) \\
	&= (\partial_{tt}- c^{2} \partial_{xx}) u + k(\partial_{tt} - c^{2}  \partial_{xx}) v \\
	&= \LL u + k\LL v
	\end{align*}
	Hal ini berakibat bahwa persamaan gelombang merupakan PDP linier. 
\end{enumerate}
\section{Persamaan Differensial Parsial}
Untuk fungsi dua peubah $u(x,y)$ bentuk umum pdp orde satu adalah:
$$ F(x,y,u(x,y),u_{x}(x,y),u_{y}(x,y))=F(x,y,u,u_x,u_y)=0,$$
Sedangkan bentuk umum PDP orde dua adalah:
$$F(x,y,u,u_{x},u_{y},u_{xx},u_{xy},u_{yy})=0.$$
Solusi dari PDP adalah fungsi $u(x,y)$ yang memenuhi persamaan diferensial tersebut, untuk suatu daerah di bidang-$xy$. 
\subsection{Persamaan Differensial Parsial Linear Orde $1$}
\subsubsection{Persamaan Transport}
\subsection{Persamaan Differensial Parsial Linear Orde $2$}


\begin{enumerate}
	\item[1.]
	
\end{enumerate}





	

$$\rho_{ij}(k)=\frac{\gamma_{ij}(k)}{\sqrt{\gamma_{ii}(0)\gamma_{jj}(0)}} \text{ dimana}$$
$$\gamma_{ij}(k) = E\left [ \left (  Z_{i,t}-\mu_{i}\right )\left ( Z_{j,t+k}-\mu_{j} \right ) \right ] = E\left [ \left (  Z_{i,t-k}-\mu_{i}\right )\left ( Z_{j,t}-\mu_{j} \right ) \right ]$$






















\end{document}