%%%%%%%%%%%%%%%%%%%%%%%%%%%%%%%%%%%%%%%%%
% Beamer Presentation
% LaTeX Template
% Version 1.0 (10/11/12)
%
% This template has been downloaded from:
% http://www.LaTeXTemplates.com
%
% License:
% CC BY-NC-SA 3.0 (http://creativecommons.org/licenses/by-nc-sa/3.0/)
%
%%%%%%%%%%%%%%%%%%%%%%%%%%%%%%%%%%%%%%%%%

%----------------------------------------------------------------------------------------
%	PACKAGES AND THEMES
%----------------------------------------------------------------------------------------

\documentclass{beamer}

\mode<presentation> {

% The Beamer class comes with a number of default slide themes
% which change the colors and layouts of slides. Below this is a list
% of all the themes, uncomment each in turn to see what they look like.

%\usetheme{default}
%\usetheme{AnnArbor}
%\usetheme{Antibes}
%\usetheme{Bergen}
%\usetheme{Berkeley}
%\usetheme{Berlin}
%\usetheme{Boadilla}
%\usetheme{CambridgeUS}
%\usetheme{Copenhagen}
%\usetheme{Darmstadt}
%\usetheme{Dresden}
%\usetheme{Frankfurt}
%\usetheme{Goettingen}
%\usetheme{Hannover}
%\usetheme{Ilmenau}
%\usetheme{JuanLesPins}
%\usetheme{Luebeck}
%\usetheme{Madrid}
%\usetheme{Malmoe}
%\usetheme{Marburg}
%\usetheme{Montpellier}
%\usetheme{PaloAlto}
%\usetheme{Pittsburgh}
%\usetheme{Rochester}
\usetheme{Singapore}
%\usetheme{Szeged}
%\usetheme{Warsaw}

% As well as themes, the Beamer class has a number of color themes
% for any slide theme. Uncomment each of these in turn to see how it
% changes the colors of your current slide theme.

%\usecolortheme{albatross}
%\usecolortheme{beaver}
%\usecolortheme{beetle}
%\usecolortheme{crane}
%\usecolortheme{dolphin}
%\usecolortheme{dove}
%\usecolortheme{fly}
%\usecolortheme{lily}
\usecolortheme{orchid}
%\usecolortheme{rose}
%\usecolortheme{seagull}
%\usecolortheme{seahorse}
%\usecolortheme{whale}
%\usecolortheme{wolverine}

%\setbeamertemplate{footline} % To remove the footer line in all slides uncomment this line
%\setbeamertemplate{footline}[page number] % To replace the footer line in all slides with a simple slide count uncomment this line

%\setbeamertemplate{navigation symbols}{} % To remove the navigation symbols from the bottom of all slides uncomment this line
}

\usepackage{wrapfig}
\usepackage{graphicx} % Allows including images
\usepackage{booktabs} % Allows the use of \toprule, \midrule and \bottomrule in tables
\usepackage{xcolor}
\newtheorem{contoh}{Contoh}
\newtheorem{definisi}{Definisi}
\newtheorem{teorema}{Teorema}
%----------------------------------------------------------------------------------------
%	TITLE PAGE
%----------------------------------------------------------------------------------------

\title[Persamaan Differensial]{Persamaan Differensial} % The short title appears at the bottom of every slide, the full title is only on the title page

\author{Uzumaki Sang Raja Kucing} % Your name
\institute[ITB] % Your institution as it will appear on the bottom of every slide, may be shorthand to save space
{
Institut Teknologi Bandung \\ % Your institution for the title page
\medskip
\textit{nagato.uzumaki17@yahoo.com} % Your email address
}
\date{\today} % Date, can be changed to a custom date

\begin{document}

\begin{frame}
\titlepage % Print the title page as the first slide
\end{frame}
\section{PDLH}
\begin{frame}{PD Linier Homogen}
Persamaan differensial biasa berorde-$\color{blue}n$ memiliki bentuk umum
\begin{align*}
\color{blue} y^{(n)}+a_1(x)y^{(n-1)}+\cdots+a_{n-1}y'+a_ny=\phi(x)
\end{align*}
Persamaan ini \textbf{linear} karena jika $\color{blue}f(x)$ dan $\color{blue}g(x)$ solusi dari persamaan diferensial tersebut maka $\color{blue}kf(x)$ dan $\color{blue}f(x)+g(x)$ juga merupakan solusi. 

Untuk $\color{blue}\phi(x)=0$ kita sebut persamaan diferensial tersebut adalah persamaan diferensial \textbf{homogen}.
\end{frame}
\subsection{Persamaan Diferensial Homogen Orde 2}
\begin{frame}{Persamaan Diferensial Homogen Orde 2}
	Misalkan persamaan diferensial berbentuk $\color{blue}y''+a_1y+a_2y=0$. \pause
	Didefinisikan \textbf{persamaan karakteristik } dari persamaan diferensial tersebut adalah $\color{blue} r^2+a_1r+a_2=0.$ \pause 
	\begin{teorema}
		Jika $\color{blue}r_1$ dan $\color{blue}r_2$ adalah akar-akar real berbeda dari persamaan karakteristik dari persamaan diferensial $\color{blue}y''+a_1y'+a_2y=0$ maka solusi umum dari persamaan diferensial tersebut adalah $$\color{blue}y(x)=c_1e^{r_1x}+c_2e^{r_2x}.$$
	\end{teorema} \pause 
\end{frame}


\subsection{Persamaan Diferensial Homogen Orde yang Lebih Tinggi}
\begin{frame}{Persamaan Diferensial Homogen Orde yang Lebih Tinggi}
Persamaan karakteristik dari persamaan diferenaial
\begin{align*}
\color{blue} y^{(n)}+a_1(x)y^{(n-1)}+\cdots+a_{n-1}y'+a_ny=0
\end{align*}
adalah 
\begin{align*}
\color{blue} r^n+a_1r^{n-1}+a_2r^{n-2}+\cdots+a_{n-1}r+a_n=0.
\end{align*}
\begin{contoh}
	Misalkan persamaan karakteristik dari suatu persamaan diferensial adalah $$\color{blue}(r-r_1)(r-r_2)^3(r-(\alpha+i\beta))(r-(\alpha-i\beta))=0$$
	maka solusi dari persamaan diferensial tersebut adalah
	\begin{align*}
	\color{blue} y(x)=c_1e^{r_1x}+(c_2+c_3x+c_4x^2)e^{r_2x}+(e^{\alpha x}(c_5\cos x+c_6\sin x))
	\end{align*}
\end{contoh}
\end{frame}

\section{PDLNH}
\begin{frame}{Persamaan Diferensial Linier Non Homogen}
	Bentuk umum persamaan linier non homogen.
	\begin{align*}
	\color{blue} y^{(n)}+a_1(x)y^{(n-1)}+\cdots+a_{n-1}y'+a_ny=\phi(x)
	\end{align*}
	dengan $\color{blue}\phi(x)\neq 0$. 
\end{frame}


\begin{frame}
%	\begin{wrapfigure}{l}{0.4\textwidth}
%		\centering
%		\includegraphics[width=0.4\textwidth]{6d.jpg}
%	\end{wrapfigure}
	\textbf{Kasus 1.} Jika $\color{blue}E^2-4B^2<0$ maka diperoleh akar-akarnya adalah $\color{blue}r_{1,2}=-\alpha\pm i\beta$. Solusi persamaan diferensialnya adalah
	$$\color{blue}y(t)=c_1e^{-\alpha t}\cos(\beta t)+c_2e^{-\alpha t}\sin(\beta t).$$
\end{frame}
\begin{frame}
%	\begin{wrapfigure}{l}{0.5\textwidth}
%		\centering
%		\includegraphics[width=0.5\textwidth]{gambar3.png}
%	\end{wrapfigure}
	\textbf{Kasus 2.} Jika $\color{blue}E^2-4B^2=0$ maka diperoleh akarnya kembar $r_{1,2}=-\alpha$ dimana $\color{blue}\alpha=\frac{E}{2}$. Solusi persamaan diferensialnya adalah
	$$\color{blue}y(t)=c_1e^{-\alpha t}+c_2te^{-\alpha t}$$
\end{frame}

\begin{frame}{Daftar Pustaka}
	\begin{thebibliography}{9}
		\bibitem{Purcell} 
		Verberg, Purcell and Ridgon, (2007). \textit{Calculus}(9$^{th}$ edition). Southern illinois university edwardsville.
		\bibitem{Soal-soal} 
		Soal-soal Tutorial MAC 2019/2020
	\end{thebibliography}
\end{frame}


\newpage
\begin{frame}
\frametitle{Sample frame title}
This is a text in second frame. 
For the sake of showing an example.

\begin{itemize}
	\item<1-> Text visible on slide 1
	\item<2-> Text visible on slide 2
	\item<3> Text visible on slide 3
	\item<4-> Text visible on slide 4
\end{itemize}

\end{frame}



\newpage
\begin{frame}
\frametitle{Sample frame title}

In this slide, some important text will be
\alert{highlighted} because it's important.
Please, don't abuse it.

\begin{block}{Remark}
Sample text
\end{block}

\begin{alertblock}{Important theorem}
Sample text in red box
\end{alertblock}

\begin{examples}
Sample text in green box. The title of the block is ``Examples".
\end{examples}
\end{frame}








\end{document} 
